% ******************************* Thesis Appendix A ********************************
\chapter{Code}
The code along with the data and model files for this project is available at a Github repository \href{https://github.com/azardilis/fa_metabolism}{here}.
Following is a brief explanation of what is contained in the files in
the repository.

\begin{table}[h]
\centering
\begin{tabular}{lp{7cm}}
\textbf{Filename} & \textbf{Description} \\
\texttt{fa.py}  &  Contains code for loading a model, executing it, and writing results to file (see Figure~\ref{fig:code}) \\[0.03cm]
\\
\texttt{fa.pyx}         &  Same as \texttt{fa.py} but in Cython for improved performance          \\[0.03cm]\\
\texttt{ml\_est.R}         &  Contains code for the two methods for
parameter inference for the basic model as described in
Section~\ref{sec:params}. Also includes loading, filtering, and
parameter calculation for the experimental dataset as presented in
Section~\ref{sec:results} \\[0.03cm] \\
\texttt{data/GCFIDrawdata.csv} & Contains the raw experimental data
with intensities for all FA products \\[0.03cm] \\
\texttt{model/fa\_elongation\_full.spstochpn} & SNOOPY \cite []
{heiner2012snoopy} model file for the basic model presented in
Section~\ref{sec:pn_implementation} \\[0.03cm] \\
\texttt{model/fa\_elongation\_full.psc} & Same as above but in the
PySCeS language. Can be loaded with the \texttt{load\_model} function in
\texttt{fa.py} \\[0.03cm] \\
\texttt{model/fa\_metabolism.spi} & SPiM version of the basic model as
described in Section~\ref{sec:picalc_implementation}. See
\href{http://research.microsoft.com/en-us/projects/spim/}{SPiM project
  website} for information on how to execute it \\[0.03cm] \\
\texttt{model/fa\_synthesis\_control.spstochpn} & SNOOPY model file for the
extended model presented in Section~\ref{sec:pn_ext_implementation}\\
\end{tabular}
\end{table}




