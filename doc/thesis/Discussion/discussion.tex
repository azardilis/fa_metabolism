\chapter{Discussion}

% **************************** Define Graphics Path **************************
\ifpdf
    \graphicspath{{Discussion/Figs/Raster/}{Discussion/Figs/PDF/}{Discussion/Figs/}}
\else
    \graphicspath{{Discussion/Figs/Vector/}{Discussion/Figs/}}
\fi

This section is divided into two logical parts. The first two
sections are commentary and critique on firstly the created model and
secondly the languages we used to capture the model. The last section
takes the form of perspectives and future directions of both the
language methodology and our developed model.

\section{Simplified model and accuracy of description}
In this work we created an abstract simplified reaction-centric model of the elongation
process that takes place in the cell to grow the CH tails of
FAs. The model we developed captures the defining high-level
characteristics of this process accurately. The developed model allows
us to observe the series of elongation steps and get distributions of
outputs of FA products that can be compared with real experimental
datasets as we have done in section Results. We have also shown how
we can tune the parameters of the model in the presence of such datasets
. The assumptions we made however
for the contrusction of the model make it biochemically
inaccurate. So while it might provide a high level view of the model
in a conceptually clear way it does not reflect the
actual biochemical process that takes place inside the cell. All the
reaction rate functions were assumed to be constant and treated as
probabilities without regarding enzymatic activity and numbers of
molecules of the
reactants. In fact all the concatenation steps of the process involve
the activity of the same enzymes so the basal reaction rates
(probability of reaction between two reactant molecules) should be the same for
all elongation steps. There must be then some other underlying
biochemical mechanism that
gives rise to the probabilistic interpretation we provided through our
model. Each concatenation step was also considered as a single
reaction instead of the four that it actually takes. This assumption does not pose a big problem for the
biochemical validity of the model if the stoichiometries and enzyme
activities participating in the reactions are considered
carefully. The other big simplification was the combination of the two
pathways acting together for the elongation of FAs, FA biosynthesis in
the cytosol
for FA lengths up to 16 and FA elongation in ER for longer FAs. The
transportation between the cytsol was not considered.

In the extended model we considered the relation between different
pathways in the metabolic machinery of the cell. We only wanted to
investigate the communication between the interfaces of the different
pathways so we represented pathways by one transition and explicitly
only the metabolites that provide the communication interface with
other pathways and most importantly the FA biosynthesis/elongation
process. So for example the entire TCA cycle and respiratory chain
responsible for the production of energy was represented as one transition with
only ATP as output as that is the point of
communication with the FA biosynthesis process. Also this interaction
was represented as a single step while in reality it goes through the
AMPK protein kinase. Again transportation between the compartments
that in this case also inlude the mitochondrion is not included and
Acetyl-CoA is represented by a single place. An interesting aspect of
our model is that we can also capture the communication between the
pathways and the environment for example by setting the rate for
glucose intake or the rate of ATP consumption.

In a way our models do not completely fall into the bottom-up or
top-down approaches to model construction. We started from some
knowledge of the system to construct the model in the first place
(top-down) but then experimental data guided some of the decisions in
the model simplification. In the end the FA biosynthesis/elongation
simplified model can provide accurate description and a mechanistic model for the reproduction of
measurable quantities but not a mechanistic model for the underlying
biochemical process. We still consider this a reaction-centric model
though despite the fact that the Petri Net transitions do not exactly
correspond to accurate biochemcical reactions. It is still
reaction-centric because the main events of the model are the
transformations of species from one form to the other.

\section{Description languages}
One of the project goals was also to investigate modelling methodology
for a suitable language to describe the reaction-centric view of lipid
metabolism. We particularly focused on Petri Nets that were the main
language used but we also provided an alternative stochastic
pi-calculus (SPiM) implementation of the basic model.

Petri Nets have an attractive graphical notation language that
captures the main
characteristics we identified in lipid metabolism that served as
motivation for this reacton-centric view: iterative conversion
processes and probabilistic decisions. There is also a very natural
correspondence between Petri Net transitions and reactions. The
behaviour of the net is described with formal operational semantics
that captures this reaction-centric view since the transition is the
main component in the evolution of the computation. Because of the
similarity with the traditional biochemical view of the system and the
static diagrams used to described it we can think of Petri Nets as a
dynamic executable version of the KEGG diagrams. Petri Nets are also
expressive enough to allow us to move up or down the abstraction
scale. For example in the extended net model we had places
corresponding to entire pathways or places that were placeholders for
the flow towards a specific pathway (\texttt{FAsyn} place in
Figure~\ref{fig:pn_ext}). The formal operational semantics have a
graphical equivalent in the form of the token game that can provide
intuition in the workings of a pathway or network. There is a long
tradition of using Petri Nets so there exist many tools to 'draw' nets
and play the token game. In this work we mainly focused on
quantitative analysis of stochastic Petri Nets but the dynamic picture
is only part of the strength of Petri Nets. Their formal specification
allows for qualititative static topological analyses that yield the
same insides as standard techniques in the field like Elementary Nodes
analysis. Another thing that we have not touched here but it is of
particular interest especially as the models grow in size is the model
checking ability we have with the use of Petri Nets. We can use model
checking techniques for example to explore qualitative and quantitive
static and dynamic properties of systems. \citet{gilbert2007unifying}
go as far as to provide a unifying framework for both qualitative and
quantitative analysis of biochemical networks captured with the Petri
Net language. It is interesting that we can use the same language to
analyse and talk about features of systems that are usually captured
with different approaches. One big drawback of Petri Nets is that they
are monolithic. Models cannot be decomposed to smaller parts in order
for example to abstract away non-important details of parts of the
process that are not of immediate interest. We did this informally in
the extended version of the model by defining entire pathways as
transitions and explicitly only the metabolites that provided the link
to other pathways. This is not captured formally though in the
language. The inherent coupling with the chemical reaction view makes
the constructed models grow linearly with the number of
reactions. This is not a big problem for metabolic pathways because of
the iterative chain of conversion that usually takes place. For
larger systems however with combinatorial interactions between species
there are scaling issues (See Future perspectives for some thoughts on
decomposability and scalability).

Stochastic pi-calculus and specifically SPiM are more detached from
the standard biochemical graphical notation. Each component of the
system is defined indepedently in contrast to the Petri Nets where the
main unit of definition are the transitions corresponding to the
reactions. Their operational reduction semantics are howerer defined in terms
of communication between the components that correspond to the
reactions. It is particularly suited to metabolism because metabolic
pathways usually involve linear series of conversions. These linear
series of conversion can be encoded with the sequential operator of
the calculus that defined state changes of a component afer
communication actions(reactions). The other characterstic we are
interested in, decisions, is also captured nicely with the choice
operator of the calculus that provides non-determinism. Another
advantage of stochastic pi-calculus over Petri Nets is the
decomposition that is inherent in the syntax/semantics. This coupled
with the fact that the model size grows linearly with the number of
species (instead of reactions) could make them more attractive for
larger networks.



\section{Future perspectives}

\subsection{Modelling and experimental validation}
In this section we present some possible future directions for our
contstructed models and experimental validation.

Using the already contstructed models as a scaffold we could expand
them to include interactions with other pathways. In this work we
concentrated on even-chain saturated FAs but these are not the only FA
products. FAs can get modified after creation not only but the
elongation pathway we considered but by adding double bonds at one
(monounsatureate) or many (polyunsaturated) places in their CH
tails. Then these FA products of different lengths and numbers of
double bond combine as building blocks to form more complex lipid
products like triacylglycerols (TAGs) and diacylglycerols (DAGs). DAGs
and TAGs contain a headgroup and two or three FAs
respectively. Following from our high level view of FA
biosynthesis/elongation we could add probabilistic further
modifications and probabilistic complex formation. Our current
modelling methoodology is suited for this kind of systems. For example
if we also had some experimental data about the frequencies of the
triplets or pairs of FAs found in TAGs and DAGs respectively we could
extend the model by making the current sinks of the system feed into
transitions for complex formation using the frequencies to infer the
probabilities. Perhaps we could start from yeast that only has
mono-unsaturated FAs to limit the combinatorial space in complex
formation \cite [] {nielsen2009systems}. 
Production and modelling the net forward direction is only part of
the story however and the corresponding catabolic processes are
equally important and the balance between them one of the major goals
of the interplay and control between different parts of the metabolic
machine. For example we could and should definitely add degradation of
FAs. Finally as an extension to our model with control we could add
other control mechanisms. The control of of Acetyl-CoA flow to FA
biosynthesis through ATP and AMPK is only one point of control. The
production of $C_{16}$ also inhibits the flow to prevent the
accumulation of FAs for example.


In order for our models to be valid in their representation of some
aspect of the natural world we need to be able to experimentally
validate them by comparing their execution to real data. We have done
that for our basic model for which we have output real experimental
data. It would be interesting to get experimental data to confirm the
control mechanism between ATP and Acetyl-CoA flow to FA formation. For
example we could use $^{13}C$ metabolite labelling to get time-series
data of the flows. This would allow us to calculate the strength of
the response to ATP change in Acetyl-CoA flow towards FA biosynthesis
and would allow us to tune the $k$ parameter we used in our extended
model for the transition from Acetyl-CoA to the \texttt{FAsyn}
place. Having experimental data across different conditions, like we
had for example for our basic model, could provide another useful
dimension to our models. In the Results section we calculated the
model parameters for different clusters of lipid product profiles. By
observing changes in the parameters across conditions we could
pinpoint areas or structures that show behavioural changes in the model. This could lead
to mechanistic understanding of changes that lead to different
conditions and disorders. This would be far more attractive with the
time-series data and the control mechanisms that appear to be the
reasons behind several metabolic disorders. Even with our high level
models we could for example calculate the strength response parameter
$k$ across conditions.


\subsection{Language methods}
As we have mentioned in previous sections Petri Nets may not be suited
to larger systems. Even for smaller system they lack the
expressiveness to capture different parts of models at different
levels of abstraction. \citet{lprogramming} propose Petri Nets with
Boundaries that can decompose Petri Nets and compositional operators
to define the interconnections between different parts. This approach
has a graphical notation where each part is enclosed in box
representing its boundaries. Each part is itself a Petri Net
consisting of an internal structure of places and transitions as we
have seen before. Every box has output ports that connect it to the
outside world. These ports can be connected with places inside the box
via normal transitions (Figure~\ref{fig:pnb}). The main operator of the algebra is the
synchronisation between these parts for communication. The researchers
have also provided a nice programming language borrowing elements from
functional programming to use along with the
algebra for Petri Nets with boundaries that I think it is important
in order for these formalisms to be adopted by practitioners. Now this
approach is quite attractive because it allows for the decomposition
of net models in an intuitive way and then communication between the
components through the synchronisation operator. In a way it borrows
some of the advantages of process-algebra models while keeping the
vivid graphical notation of Petri Nets. For our purposes of biological
modelling that would mean the different interacting parts could
different pathways or different compartments. If for example we are
only interested in one pathway we could define that part in detail
while define only the interfaces interacting with the outside world
for other pathways. This is what we have done informally in our
extended model but this approach makes it formal. This approach
however has as its main motivation behind the decomposition the
minimisation of state-space for model checking. It will need some
further work for it to be executable and useful for biological
modelling. I am just presenting it here because I think it is a step
in the right direction.

\begin{figure}[htbp!]
\centering
\includegraphics[width=1.0\textwidth]{pnb}
\caption[Petri Nets with boundaries]{Petri Nets with boundaries. On
  the left two composing parts of the net. Each has its own internal
  net structure and transitions. Some of the places can communicate
  via transitions withe the outports of that part(squares at the edge
  of the box). Communication between the two parts happens through
  synchronisation of the outports of the two(picture on the right).}
\label{fig:pnb}
\end{figure}

This is part of a more general attempt to capture hierarchy in
these minimilastic languages for distributed computing. In Petri Nets
with boundaries we can talk at two levels, the local level of the
transitions and state changes inside a box and then the communication
through synchronisation between boxes. Another attempt that falls in
the same category is bigraphs that also has a graphical notation and
the ability to nest structures \cite []
{jensen2004bigraphs}. \citet{degasperi2013process} proposed a process
algebra with hooks that attempts to provide modelling capabilities for
multi-level biological systems and communication operators between the
levels. I think that these are steps in the right direction because we
are interested in qualities of biological systems that naturally fall
on different levels. We should be able to express the higher level
qualities in terms of the lower level qualities. That way we could
possible have multi-level stratified design of biological systems akin
to computer programs organisation and abstraction \cite [] {abelson1987lisp}.

