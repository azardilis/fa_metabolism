\chapter{Discussion}

% **************************** Define Graphics Path **************************
\ifpdf
    \graphicspath{{Discussion/Figs/Raster/}{Discussion/Figs/PDF/}{Discussion/Figs/}}
\else
    \graphicspath{{Discussion/Figs/Vector/}{Discussion/Figs/}}
\fi

This section is divided into two logical parts. The first two
sections are commentary and critique on firstly the created model and
secondly the languages we used to capture the model. The last section
takes the form of perspectives and future directions of both the
language methodology and our developed model.

\section{Simplified model and accuracy of description}
In this work we created an abstract simplified reaction-centric model of the elongation
process that takes place in the cell to grow the CH tails of
FAs. The model we developed captures the defining high-level
characteristics of this process accurately. The developed model allows
us to observe the series of elongation steps and get distributions of
outputs of FA products that can be compared with real experimental
datasets as we have done in section Results. We have also shown how
we can tune the parameters of the model in the presence of such datasets
. The assumptions we made however
for the contrusction of the model make it biochemically
inaccurate. So while it might provide a high level view of the model
in a conceptually clear way it does not reflect the
actual biochemical process that takes place inside the cell. All the
reaction rate functions were assumed to be constant and treated as
probabilities without regarding enzymatic activity and numbers of
molecules of the
reactants. In fact all the concatenation steps of the process involve
the activity of the same enzymes so the basal reaction rates
(probability of reaction between two reactant molecules) should be the same for
all elongation steps. There must be then some other underlying
biochemical mechanism that
gives rise to the probabilistic interpretation we provided through our
model. Each concatenation step was also considered as a single
reaction instead of the four that it actually takes. This assumption does not pose a big problem for the
biochemical validity of the model if the stoichiometries and enzyme
activities participating in the reactions are considered
carefully. The other big simplification was the combination of the two
pathways acting together for the elongation of FAs, FA biosynthesis in
the cytosol
for FA lengths up to 16 and FA elongation in ER for longer FAs. The
transportation between the cytsol was not considered.

In the extended model we considered the relation between different
pathways in the metabolic machinery of the cell. We only wanted to
investigate the communication between the interfaces of the different
pathways so we represented pathways by one transition and explicitly
only the metabolites that provide the communication interface with
other pathways and most importantly the FA biosynthesis/elongation
process. So for example the entire TCA cycle and respiratory chain
responsible for the production was represented as one transition with
only ATP explicitly modelled as place because that is the point of
communication with the FA biosynthesis process. Also this interaction
was represented as single step while in reality it goes through the
AMPK protein kinase. Again transportation between the compartments
that in this case also inlude the mitochondrion is not included and
Acetyl-CoA is represented by a single place. An interesting aspect of
our model is that we can also capture the communication between the
pathways and the environment for example by setting the rate for
glucose intake or the rate of ATP consumption.

In a way our models do not completely fall into the bottom-up or
top-down approaches to model contstruction. We started from some
knowledge of the system to construct the model in the first place
(top-down) but then experimental data guided some of the decisions in
the model simplification. In the end the FA biosynthesis/elongation
simplified model can provide accurate description and a mechanistic model for the reproduction of
measurable quantities but not a mechanistic model for the underlying
biochemical process. We still consider this a reaction-centric model
though despite the fact that the Petri Net transitions do not exactly
correspond to accurate biochemcical reactions. It is still
reaction-centric because the main events of the model are the
transformations of species from one form to the other.

\section{Description languages}
One of the project goals was also to investigate modelling methodology
for a suitable language to describe the reaction-centric view of lipid
metabolism. We particularly focused on Petri Nets that were the main
language used but we also provided an alternative stochastic
pi-calculus (SPiM) implementation of the basic model.

\textbf{The following text in this Dicsussion is just extended notes(not complete yet):}
Petri Nets nicely captures the iterative process and probabilistic
decisions we identified as the main characteristics and motivations
for this reaction-centric view. This is also inherent in the intutive
graphical notation. Good tools, nice dyanmic view with token game
which is defined by formal operational semantics. Also flexible allows
to model any kind of communication not only physical reactions but
also our more high level interactions. Not done here but
could also do static topological analysis along with quantitative
analysis. For bigger networks could also employ model checking
technqiues since the language is formally defined. Petri Nets are
however very monolithic. Need to capture entire system in one
model. For example can's abstract away, at least not formally, details of one pathway and
provide only the communication links with other parts. (See future
perspectives for some thoughts on this)

Stochastic pi-calculus provides the reaction centric view we want and
the decisions that are so important. Both from the syntax but also
with the nice graphical notation. Non a straightforward correspondence
between chemical reactions and pi-calculus constructs. Metabolites
defined independently so the size of model grows linearly with number
of species not number of reactions. Could be better for larger
nets. Also this definition allows for decomposability and modularity
which can make this approach scale better.

\section{Future perspectives}

\subsection{Language methods}
Petri Nets with boundaries to create modules of interaction with some
interfaces between the modules? This is exactly what we have done
informally in our extended model. This approach combines the
advantages of Petri Nets and pi-calculus. Could be a good
solution. Sobocinski paper looks promising. The composition operator
between bounded PNs could for example be the 'cross-talk' between
pathways. Modularity -> allows us to define just the interface and not
the actual implementation of pathways we are not interested at the
moment. Could also improve collaborations. One group works on one
pathway in detail, another one some other and they only have to worry
about the interfaces.

Would be good to have a language that we could use to abstract away
unwanted details and be able to defined higher level qualities of the
system in terms of lower level details. Stratified design like
computer programs! Some attempts: process algebras with hooks from
Glasgow and bigraphs from Milner.

\subsection{Modelling and experimental validation}
Evaluate our models and experiments to maybe calc the parameters of
the extenedn model as well. Time-series measurements so we can see
control/strength of control.

Add other modifications and cross-talk between pathways. For example
DAGs, TAGs combinatorial problems. Could also measure these somehow.

Definetely add degradation. Balance is important.

The regulation of FA biosynthesis through AMPK is just one point of
control. Could also add others.

With a more biochemically valid model we could also something similar
to what we have done in the Results section. See the differences
between disease/non-disease. With a biochemical valid model could
identify mechanisms and misregulations.






